\PassOptionsToPackage{unicode=true}{hyperref} % options for packages loaded elsewhere
\PassOptionsToPackage{hyphens}{url}
%
\documentclass[
]{article}
\usepackage{lmodern}
\usepackage{amssymb,amsmath}
\usepackage{ifxetex,ifluatex}
\ifnum 0\ifxetex 1\fi\ifluatex 1\fi=0 % if pdftex
  \usepackage[T1]{fontenc}
  \usepackage[utf8]{inputenc}
  \usepackage{textcomp} % provides euro and other symbols
\else % if luatex or xelatex
  \usepackage{unicode-math}
  \defaultfontfeatures{Scale=MatchLowercase}
  \defaultfontfeatures[\rmfamily]{Ligatures=TeX,Scale=1}
\fi
% use upquote if available, for straight quotes in verbatim environments
\IfFileExists{upquote.sty}{\usepackage{upquote}}{}
\IfFileExists{microtype.sty}{% use microtype if available
  \usepackage[]{microtype}
  \UseMicrotypeSet[protrusion]{basicmath} % disable protrusion for tt fonts
}{}
\makeatletter
\@ifundefined{KOMAClassName}{% if non-KOMA class
  \IfFileExists{parskip.sty}{%
    \usepackage{parskip}
  }{% else
    \setlength{\parindent}{0pt}
    \setlength{\parskip}{6pt plus 2pt minus 1pt}}
}{% if KOMA class
  \KOMAoptions{parskip=half}}
\makeatother
\usepackage{xcolor}
\IfFileExists{xurl.sty}{\usepackage{xurl}}{} % add URL line breaks if available
\IfFileExists{bookmark.sty}{\usepackage{bookmark}}{\usepackage{hyperref}}
\hypersetup{
  pdftitle={QAQC of the CINP ASSP 1994 - 2018 CPUE data},
  pdfauthor={EKelsey},
  pdfborder={0 0 0},
  breaklinks=true}
\urlstyle{same}  % don't use monospace font for urls
\usepackage[margin=1in]{geometry}
\usepackage{color}
\usepackage{fancyvrb}
\newcommand{\VerbBar}{|}
\newcommand{\VERB}{\Verb[commandchars=\\\{\}]}
\DefineVerbatimEnvironment{Highlighting}{Verbatim}{commandchars=\\\{\}}
% Add ',fontsize=\small' for more characters per line
\usepackage{framed}
\definecolor{shadecolor}{RGB}{248,248,248}
\newenvironment{Shaded}{\begin{snugshade}}{\end{snugshade}}
\newcommand{\AlertTok}[1]{\textcolor[rgb]{0.94,0.16,0.16}{#1}}
\newcommand{\AnnotationTok}[1]{\textcolor[rgb]{0.56,0.35,0.01}{\textbf{\textit{#1}}}}
\newcommand{\AttributeTok}[1]{\textcolor[rgb]{0.77,0.63,0.00}{#1}}
\newcommand{\BaseNTok}[1]{\textcolor[rgb]{0.00,0.00,0.81}{#1}}
\newcommand{\BuiltInTok}[1]{#1}
\newcommand{\CharTok}[1]{\textcolor[rgb]{0.31,0.60,0.02}{#1}}
\newcommand{\CommentTok}[1]{\textcolor[rgb]{0.56,0.35,0.01}{\textit{#1}}}
\newcommand{\CommentVarTok}[1]{\textcolor[rgb]{0.56,0.35,0.01}{\textbf{\textit{#1}}}}
\newcommand{\ConstantTok}[1]{\textcolor[rgb]{0.00,0.00,0.00}{#1}}
\newcommand{\ControlFlowTok}[1]{\textcolor[rgb]{0.13,0.29,0.53}{\textbf{#1}}}
\newcommand{\DataTypeTok}[1]{\textcolor[rgb]{0.13,0.29,0.53}{#1}}
\newcommand{\DecValTok}[1]{\textcolor[rgb]{0.00,0.00,0.81}{#1}}
\newcommand{\DocumentationTok}[1]{\textcolor[rgb]{0.56,0.35,0.01}{\textbf{\textit{#1}}}}
\newcommand{\ErrorTok}[1]{\textcolor[rgb]{0.64,0.00,0.00}{\textbf{#1}}}
\newcommand{\ExtensionTok}[1]{#1}
\newcommand{\FloatTok}[1]{\textcolor[rgb]{0.00,0.00,0.81}{#1}}
\newcommand{\FunctionTok}[1]{\textcolor[rgb]{0.00,0.00,0.00}{#1}}
\newcommand{\ImportTok}[1]{#1}
\newcommand{\InformationTok}[1]{\textcolor[rgb]{0.56,0.35,0.01}{\textbf{\textit{#1}}}}
\newcommand{\KeywordTok}[1]{\textcolor[rgb]{0.13,0.29,0.53}{\textbf{#1}}}
\newcommand{\NormalTok}[1]{#1}
\newcommand{\OperatorTok}[1]{\textcolor[rgb]{0.81,0.36,0.00}{\textbf{#1}}}
\newcommand{\OtherTok}[1]{\textcolor[rgb]{0.56,0.35,0.01}{#1}}
\newcommand{\PreprocessorTok}[1]{\textcolor[rgb]{0.56,0.35,0.01}{\textit{#1}}}
\newcommand{\RegionMarkerTok}[1]{#1}
\newcommand{\SpecialCharTok}[1]{\textcolor[rgb]{0.00,0.00,0.00}{#1}}
\newcommand{\SpecialStringTok}[1]{\textcolor[rgb]{0.31,0.60,0.02}{#1}}
\newcommand{\StringTok}[1]{\textcolor[rgb]{0.31,0.60,0.02}{#1}}
\newcommand{\VariableTok}[1]{\textcolor[rgb]{0.00,0.00,0.00}{#1}}
\newcommand{\VerbatimStringTok}[1]{\textcolor[rgb]{0.31,0.60,0.02}{#1}}
\newcommand{\WarningTok}[1]{\textcolor[rgb]{0.56,0.35,0.01}{\textbf{\textit{#1}}}}
\usepackage{graphicx,grffile}
\makeatletter
\def\maxwidth{\ifdim\Gin@nat@width>\linewidth\linewidth\else\Gin@nat@width\fi}
\def\maxheight{\ifdim\Gin@nat@height>\textheight\textheight\else\Gin@nat@height\fi}
\makeatother
% Scale images if necessary, so that they will not overflow the page
% margins by default, and it is still possible to overwrite the defaults
% using explicit options in \includegraphics[width, height, ...]{}
\setkeys{Gin}{width=\maxwidth,height=\maxheight,keepaspectratio}
\setlength{\emergencystretch}{3em}  % prevent overfull lines
\providecommand{\tightlist}{%
  \setlength{\itemsep}{0pt}\setlength{\parskip}{0pt}}
\setcounter{secnumdepth}{-2}
% Redefines (sub)paragraphs to behave more like sections
\ifx\paragraph\undefined\else
  \let\oldparagraph\paragraph
  \renewcommand{\paragraph}[1]{\oldparagraph{#1}\mbox{}}
\fi
\ifx\subparagraph\undefined\else
  \let\oldsubparagraph\subparagraph
  \renewcommand{\subparagraph}[1]{\oldsubparagraph{#1}\mbox{}}
\fi

% set default figure placement to htbp
\makeatletter
\def\fps@figure{htbp}
\makeatother


\title{QAQC of the CINP ASSP 1994 - 2018 CPUE data}
\author{EKelsey}
\date{4/14/2020}

\begin{document}
\maketitle

Load libraries

\begin{Shaded}
\begin{Highlighting}[]
\KeywordTok{library}\NormalTok{(dplyr)}
\end{Highlighting}
\end{Shaded}

\begin{verbatim}
## Warning: package 'dplyr' was built under R version 3.6.3
\end{verbatim}

\begin{verbatim}
## 
## Attaching package: 'dplyr'
\end{verbatim}

\begin{verbatim}
## The following objects are masked from 'package:stats':
## 
##     filter, lag
\end{verbatim}

\begin{verbatim}
## The following objects are masked from 'package:base':
## 
##     intersect, setdiff, setequal, union
\end{verbatim}

\begin{Shaded}
\begin{Highlighting}[]
\KeywordTok{library}\NormalTok{(ggplot2)}
\end{Highlighting}
\end{Shaded}

\begin{verbatim}
## Warning: package 'ggplot2' was built under R version 3.6.3
\end{verbatim}

Load data and add columns neccessary for QAQC:

\begin{Shaded}
\begin{Highlighting}[]
\NormalTok{metadata <-}\StringTok{ }\KeywordTok{read.csv}\NormalTok{(}\StringTok{'~/WERC-SC/ASSP_share/ASSP_4_metadata_CPUE_20200413.csv'}\NormalTok{) }\OperatorTok\StringTok{ }
\StringTok{  }\KeywordTok{mutate_at}\NormalTok{(}\KeywordTok{c}\NormalTok{(}\StringTok{"App_sunset"}\NormalTok{, }\StringTok{"std_ending"}\NormalTok{), }\DataTypeTok{.funs =} \OperatorTok{~}\KeywordTok{as.POSIXct}\NormalTok{(., }\DataTypeTok{format=}\StringTok{"%m/%d/%Y %H:%M"}\NormalTok{)) }\OperatorTok\StringTok{ }
\StringTok{  }\KeywordTok{mutate_at}\NormalTok{(}\KeywordTok{c}\NormalTok{(}\StringTok{"net_open_1"}\NormalTok{, }\StringTok{"net_close_1"}\NormalTok{, }\StringTok{"net_open_2"}\NormalTok{, }\StringTok{"net_close_2"}\NormalTok{, }\StringTok{"net_open_3"}\NormalTok{,}
              \StringTok{"net_close_3"}\NormalTok{, }\StringTok{"net_open_4"}\NormalTok{, }\StringTok{"net_close_4"}\NormalTok{, }\StringTok{"net_open_5"}\NormalTok{, }\StringTok{"net_close_5"}\NormalTok{),}
            \DataTypeTok{.funs =} \OperatorTok{~}\KeywordTok{as.POSIXct}\NormalTok{(., }\DataTypeTok{format=}\StringTok{"%Y-%m-%d %H:%M:%S"}\NormalTok{)) }\OperatorTok
\StringTok{  }\KeywordTok{mutate_at}\NormalTok{(}\KeywordTok{c}\NormalTok{(}\StringTok{"App_sunset"}\NormalTok{, }\StringTok{"std_ending"}\NormalTok{, }\StringTok{"net_open_1"}\NormalTok{, }\StringTok{"net_close_1"}\NormalTok{), }
            \DataTypeTok{.funs =} \KeywordTok{list}\NormalTok{(}\DataTypeTok{time =} \OperatorTok{~}\StringTok{ }\NormalTok{hms}\OperatorTok{::}\KeywordTok{as_hms}\NormalTok{(.))) }\OperatorTok\StringTok{ }
\StringTok{  }\KeywordTok{mutate}\NormalTok{(}\DataTypeTok{CPUE_ratio =}\NormalTok{ CPUEstd}\OperatorTok{/}\NormalTok{CPUEraw,}
         \DataTypeTok{month =} \KeywordTok{as.character}\NormalTok{(month)) }\OperatorTok\StringTok{ }
\StringTok{  }\KeywordTok{filter}\NormalTok{(}\OtherTok{TRUE}\NormalTok{)}

\NormalTok{catches <-}\StringTok{ }\KeywordTok{read.csv}\NormalTok{(}\StringTok{'~/WERC-SC/ASSP_share/ASSP_4_catches_BANDING_20200414.csv'}\NormalTok{) }\OperatorTok\StringTok{ }
\StringTok{  }\KeywordTok{mutate_at}\NormalTok{(}\KeywordTok{c}\NormalTok{(}\StringTok{"App_sunset"}\NormalTok{, }\StringTok{"std_ending"}\NormalTok{, }\StringTok{"capture_time"}\NormalTok{, }\StringTok{"release_time"}\NormalTok{),}
            \DataTypeTok{.funs =} \OperatorTok{~}\KeywordTok{as.POSIXct}\NormalTok{(., }\DataTypeTok{format=}\StringTok{"%Y-%m-%d %H:%M:%S"}\NormalTok{)) }\OperatorTok\StringTok{ }
\StringTok{  }\KeywordTok{filter}\NormalTok{(species }\OperatorTok{==}\StringTok{ "ASSP"}\NormalTok{,}
\NormalTok{         catchPastSS }\OperatorTok{>}\StringTok{ }\DecValTok{0}\NormalTok{, catchPastSS }\OperatorTok{<}\StringTok{ }\DecValTok{600}\NormalTok{)}
\end{Highlighting}
\end{Shaded}

\hypertarget{time-and-mistnetting-effort}{%
\section{Time and Mistnetting
Effort}\label{time-and-mistnetting-effort}}

\hypertarget{graphical-check-of-app_sunset}{%
\subsection{Graphical check of
App\_sunset}\label{graphical-check-of-app_sunset}}

\begin{Shaded}
\begin{Highlighting}[]
\KeywordTok{ggplot}\NormalTok{(metadata, }\KeywordTok{aes}\NormalTok{(month, App_sunset_time)) }\OperatorTok{+}
\StringTok{  }\KeywordTok{geom_point}\NormalTok{(}\KeywordTok{aes}\NormalTok{(}\DataTypeTok{color =}\NormalTok{ year)) }\OperatorTok{+}
\StringTok{  }\KeywordTok{scale_color_gradient}\NormalTok{(}\DataTypeTok{low=}\StringTok{"purple"}\NormalTok{, }\DataTypeTok{high=}\StringTok{"orange"}\NormalTok{) }\OperatorTok{+}
\StringTok{  }\KeywordTok{theme_bw}\NormalTok{()}
\end{Highlighting}
\end{Shaded}

\includegraphics{ASSP_CPUE_metadata_QAQC_files/figure-latex/unnamed-chunk-3-1.pdf}
.

\begin{quote}
This graph shows the time of apparent sunset for netting sessions each
month. The range and timing for that time of year is as we would expect.
Thus we conclude that the suncalc function was used effectively to get
the sunset times associated with each mistnetting session.
\end{quote}

\hypertarget{graphical-check-of-std_ending}{%
\subsection{Graphical check of
Std\_ending}\label{graphical-check-of-std_ending}}

\hypertarget{plotted-by-month}{%
\subsubsection{Plotted by month}\label{plotted-by-month}}

\begin{Shaded}
\begin{Highlighting}[]
\KeywordTok{ggplot}\NormalTok{(metadata, }\KeywordTok{aes}\NormalTok{(month, std_ending_time)) }\OperatorTok{+}
\StringTok{  }\KeywordTok{geom_point}\NormalTok{(}\KeywordTok{aes}\NormalTok{(}\DataTypeTok{color =}\NormalTok{ year)) }\OperatorTok{+}
\StringTok{  }\KeywordTok{scale_color_gradient}\NormalTok{(}\DataTypeTok{low=}\StringTok{"black"}\NormalTok{, }\DataTypeTok{high=}\StringTok{"light blue"}\NormalTok{) }\OperatorTok{+}
\StringTok{  }\KeywordTok{theme_bw}\NormalTok{()}
\end{Highlighting}
\end{Shaded}

\includegraphics{ASSP_CPUE_metadata_QAQC_files/figure-latex/unnamed-chunk-4-1.pdf}
.

\begin{quote}
This graph shows the time of standard ending (5.3 hours after sunset)
for netting sessions each month. The range and timing of the standard
ending track with sunset time as we would expect.
\end{quote}

\hypertarget{summarize-net_open-and-net_close}{%
\section{Summarize net\_open and
net\_close}\label{summarize-net_open-and-net_close}}

\begin{Shaded}
\begin{Highlighting}[]
\CommentTok{# first (and usually only) net open time}
\KeywordTok{summary}\NormalTok{(}\KeywordTok{as.POSIXct}\NormalTok{(metadata}\OperatorTok{$}\NormalTok{net_open_}\DecValTok{1}\NormalTok{_time))}
\end{Highlighting}
\end{Shaded}

\begin{verbatim}
##                  Min.               1st Qu.                Median 
## "1970-01-01 00:00:00" "1970-01-01 20:45:00" "1970-01-01 21:02:30" 
##                  Mean               3rd Qu.                  Max. 
## "1970-01-01 20:33:23" "1970-01-01 21:36:00" "1970-01-01 23:35:00" 
##                  NA's 
##                  "22"
\end{verbatim}

\begin{Shaded}
\begin{Highlighting}[]
\CommentTok{# first (and usually only) net close time}
\KeywordTok{summary}\NormalTok{(}\KeywordTok{as.POSIXct}\NormalTok{(metadata}\OperatorTok{$}\NormalTok{net_close_}\DecValTok{1}\NormalTok{_time))}
\end{Highlighting}
\end{Shaded}

\begin{verbatim}
##                  Min.               1st Qu.                Median 
## "1970-01-01 00:00:00" "1970-01-01 01:24:00" "1970-01-01 02:00:00" 
##                  Mean               3rd Qu.                  Max. 
## "1970-01-01 03:37:47" "1970-01-01 02:17:30" "1970-01-01 23:59:00" 
##                  NA's 
##                  "21"
\end{verbatim}

\begin{quote}
This is not a perfect way to summarize net open and close times because
the ``summarize'' function doesn't recognize times across midnight here.
But, by looking at the median and mean, we can tell that net open and
close times are usually what we would expect, with a few late/early
nights thrown in.
\end{quote}

\hypertarget{total-mistnetting-minutes-per-session}{%
\subsection{Total mistnetting minutes per
session}\label{total-mistnetting-minutes-per-session}}

\begin{Shaded}
\begin{Highlighting}[]
\KeywordTok{library}\NormalTok{(ggplot2)}
\KeywordTok{ggplot}\NormalTok{(metadata, }\KeywordTok{aes}\NormalTok{(min)) }\OperatorTok{+}
\StringTok{  }\KeywordTok{geom_histogram}\NormalTok{(}\DataTypeTok{binwidth =} \DecValTok{10}\NormalTok{) }\OperatorTok{+}
\StringTok{  }\KeywordTok{theme_bw}\NormalTok{()}
\end{Highlighting}
\end{Shaded}

\begin{verbatim}
## Warning: Removed 22 rows containing non-finite values (stat_bin).
\end{verbatim}

\includegraphics{ASSP_CPUE_metadata_QAQC_files/figure-latex/unnamed-chunk-6-1.pdf}

\begin{Shaded}
\begin{Highlighting}[]
\CommentTok{# summary of total mistnetting minutes}
\KeywordTok{summary}\NormalTok{(metadata}\OperatorTok{$}\NormalTok{min)}
\end{Highlighting}
\end{Shaded}

\begin{verbatim}
##    Min. 1st Qu.  Median    Mean 3rd Qu.    Max.    NA's 
##    56.0   214.0   293.0   280.3   316.0  1022.0      22
\end{verbatim}

\begin{quote}
The graph and summary stats above show minutes calcuated for each
netting session. We want to check that minutes were added correctly
across multiple open/close sessions and also that minutes were added
accurately across midnight. It looks like minutes were not added
accurately across midnight on four occasions (the outliers to the far
right)
\end{quote}

\hypertarget{total-mistnetting-standard-minutes-per-session}{%
\subsection{Total mistnetting standard minutes per
session}\label{total-mistnetting-standard-minutes-per-session}}

\hypertarget{from-start-until-end-or-standard-ending-whichever-came-first}{%
\subsubsection{from start until end or standard ending, whichever came
first}\label{from-start-until-end-or-standard-ending-whichever-came-first}}

\begin{Shaded}
\begin{Highlighting}[]
\KeywordTok{ggplot}\NormalTok{(metadata, }\KeywordTok{aes}\NormalTok{(min_std)) }\OperatorTok{+}
\StringTok{  }\KeywordTok{geom_histogram}\NormalTok{(}\DataTypeTok{binwidth =} \DecValTok{10}\NormalTok{) }\OperatorTok{+}
\StringTok{  }\KeywordTok{theme_bw}\NormalTok{()}
\end{Highlighting}
\end{Shaded}

\begin{verbatim}
## Warning: Removed 22 rows containing non-finite values (stat_bin).
\end{verbatim}

\includegraphics{ASSP_CPUE_metadata_QAQC_files/figure-latex/unnamed-chunk-7-1.pdf}

\begin{Shaded}
\begin{Highlighting}[]
\CommentTok{# summary of mistnetting minutes cut to standard ending time}
\KeywordTok{summary}\NormalTok{(metadata}\OperatorTok{$}\NormalTok{min_std)}
\end{Highlighting}
\end{Shaded}

\begin{verbatim}
##    Min. 1st Qu.  Median    Mean 3rd Qu.    Max.    NA's 
##     0.0   189.0   239.6   223.0   267.3   317.1      22
\end{verbatim}

\begin{quote}
The above graph and summary stats show the total number of mintues for
the standardized session (from net open to net close or standard ending
whichever came first). The standard ending is 5.3 hours (318 minutes)
after sunset. Nets opened sometime after sunset.
\end{quote}

\hypertarget{compare-min-vs.-min_std-for-each-session}{%
\subsection{Compare min vs.~min\_std for each
session}\label{compare-min-vs.-min_std-for-each-session}}

\begin{Shaded}
\begin{Highlighting}[]
\KeywordTok{library}\NormalTok{(ggplot2)}
\KeywordTok{ggplot}\NormalTok{(metadata, }\KeywordTok{aes}\NormalTok{(min, min_std)) }\OperatorTok{+}
\StringTok{  }\KeywordTok{geom_point}\NormalTok{(}\KeywordTok{aes}\NormalTok{(}\DataTypeTok{color =}\NormalTok{ Flagged_Y.N)) }\OperatorTok{+}
\StringTok{  }\KeywordTok{geom_abline}\NormalTok{(}\DataTypeTok{intercept =} \DecValTok{0}\NormalTok{, }\DataTypeTok{slope =} \DecValTok{1}\NormalTok{, }\DataTypeTok{color =} \StringTok{"blue"}\NormalTok{) }\OperatorTok{+}
\StringTok{  }\KeywordTok{geom_hline}\NormalTok{(}\DataTypeTok{yintercept =} \DecValTok{318}\NormalTok{, }\DataTypeTok{color =} \StringTok{"red"}\NormalTok{) }\OperatorTok{+}
\StringTok{  }\KeywordTok{xlim}\NormalTok{(}\DecValTok{0}\NormalTok{,}\DecValTok{1050}\NormalTok{) }\OperatorTok{+}\StringTok{ }\KeywordTok{ylim}\NormalTok{(}\DecValTok{0}\NormalTok{, }\DecValTok{550}\NormalTok{) }\OperatorTok{+}
\StringTok{  }\KeywordTok{theme_bw}\NormalTok{()}
\end{Highlighting}
\end{Shaded}

\begin{verbatim}
## Warning: Removed 22 rows containing missing values (geom_point).
\end{verbatim}

\includegraphics{ASSP_CPUE_metadata_QAQC_files/figure-latex/unnamed-chunk-8-1.pdf}
.

\begin{quote}
This plot of minutes vs.~standardized mintues (before the 5.3 hour
standardized ending). Blue line = slope of 1. Red line = standard
ending. Here we can make sure standardized net open minutes is equal or
less the total number of minutes and sunset to standard ending of 5.3
hours (318 minutes). Red points = data that has been flagged due to
inconsistencies in data entry. It does not appear that the reason these
entries were flagged effects net minutes.
\end{quote}

\hypertarget{assp}{%
\section{ASSP}\label{assp}}

\hypertarget{histogram-of-total-assp-caught-per-session}{%
\subsection{Histogram of total ASSP caught per
session}\label{histogram-of-total-assp-caught-per-session}}

\begin{Shaded}
\begin{Highlighting}[]
\KeywordTok{ggplot}\NormalTok{(metadata, }\KeywordTok{aes}\NormalTok{(ASSP)) }\OperatorTok{+}
\StringTok{  }\KeywordTok{geom_histogram}\NormalTok{(}\DataTypeTok{binwidth =} \DecValTok{1}\NormalTok{) }\OperatorTok{+}
\StringTok{  }\KeywordTok{theme_bw}\NormalTok{()}
\end{Highlighting}
\end{Shaded}

\begin{verbatim}
## Warning: Removed 27 rows containing non-finite values (stat_bin).
\end{verbatim}

\includegraphics{ASSP_CPUE_metadata_QAQC_files/figure-latex/unnamed-chunk-9-1.pdf}

\begin{Shaded}
\begin{Highlighting}[]
\CommentTok{# summary of ASSP catches}
\KeywordTok{summary}\NormalTok{(metadata}\OperatorTok{$}\NormalTok{ASSP)}
\end{Highlighting}
\end{Shaded}

\begin{verbatim}
##    Min. 1st Qu.  Median    Mean 3rd Qu.    Max.    NA's 
##    1.00    8.00   14.00   17.18   23.00   56.00      27
\end{verbatim}

\begin{quote}
The graph and summary stats above show the distribution of total numbers
of ASSP caught per session.
\end{quote}

\hypertarget{histogram-of-total-assp-caught-per-standardized-session}{%
\subsection{Histogram of total ASSP caught per standardized
session}\label{histogram-of-total-assp-caught-per-standardized-session}}

\begin{Shaded}
\begin{Highlighting}[]
\KeywordTok{ggplot}\NormalTok{(metadata, }\KeywordTok{aes}\NormalTok{(ASSPstd)) }\OperatorTok{+}
\StringTok{  }\KeywordTok{geom_histogram}\NormalTok{(}\DataTypeTok{binwidth =} \DecValTok{1}\NormalTok{) }\OperatorTok{+}
\StringTok{  }\KeywordTok{theme_bw}\NormalTok{()}
\end{Highlighting}
\end{Shaded}

\begin{verbatim}
## Warning: Removed 27 rows containing non-finite values (stat_bin).
\end{verbatim}

\includegraphics{ASSP_CPUE_metadata_QAQC_files/figure-latex/unnamed-chunk-10-1.pdf}

\begin{Shaded}
\begin{Highlighting}[]
\CommentTok{# summary of standardized ASSP catches}
\KeywordTok{summary}\NormalTok{(metadata}\OperatorTok{$}\NormalTok{ASSPstd)}
\end{Highlighting}
\end{Shaded}

\begin{verbatim}
##    Min. 1st Qu.  Median    Mean 3rd Qu.    Max.    NA's 
##    0.00    7.00   12.00   14.03   19.00   45.00      27
\end{verbatim}

\begin{quote}
The graph and summary stats above show the distribution of total numbers
of ASSP caught before standard ending or net close, whichever came
first. This distribution is more constrained than the one above, which
is what we would expect with the standard ending cutoff. Next we will
look into what else could effect the number of birds caught
\end{quote}

\hypertarget{number-of-assp-caught-in-relation-to-net-size}{%
\subsection{number of ASSP caught in relation to net
size}\label{number-of-assp-caught-in-relation-to-net-size}}

\begin{Shaded}
\begin{Highlighting}[]
\NormalTok{netUse <-}\StringTok{ }\NormalTok{metadata }\OperatorTok\StringTok{ }
\StringTok{  }\KeywordTok{group_by}\NormalTok{(Net_dim) }\OperatorTok\StringTok{ }
\StringTok{  }\KeywordTok{tally}\NormalTok{()}

\KeywordTok{ggplot}\NormalTok{(metadata, }\KeywordTok{aes}\NormalTok{(Net_dim, ASSPstd)) }\OperatorTok{+}
\StringTok{  }\KeywordTok{geom_boxplot}\NormalTok{() }\OperatorTok{+}
\StringTok{  }\KeywordTok{geom_text}\NormalTok{(}\DataTypeTok{data =}\NormalTok{ netUse,}
            \KeywordTok{aes}\NormalTok{(Net_dim, }\OtherTok{Inf}\NormalTok{, }\DataTypeTok{label =}\NormalTok{ n), }\DataTypeTok{vjust =} \DecValTok{1}\NormalTok{) }\OperatorTok{+}
\StringTok{  }\KeywordTok{theme_bw}\NormalTok{()}
\end{Highlighting}
\end{Shaded}

\begin{verbatim}
## Warning: Removed 27 rows containing non-finite values (stat_boxplot).
\end{verbatim}

\includegraphics{ASSP_CPUE_metadata_QAQC_files/figure-latex/unnamed-chunk-11-1.pdf}
.

\begin{quote}
The graph above shows the number of ASSP caught within the standardized
period in relation to net dimensions. Number at top of each box plot =
sample size. The size of the net doesn't seem to effect the number of
birds caught in a specific night. Below we will explore this effect with
number of catches standardized to effort.
\end{quote}

\hypertarget{comparison-of-assp-vs-asspstd}{%
\subsection{comparison of ASSP vs
ASSPstd}\label{comparison-of-assp-vs-asspstd}}

\begin{Shaded}
\begin{Highlighting}[]
\KeywordTok{ggplot}\NormalTok{(metadata, }\KeywordTok{aes}\NormalTok{(ASSP, ASSPstd)) }\OperatorTok{+}
\StringTok{  }\KeywordTok{geom_point}\NormalTok{(}\KeywordTok{aes}\NormalTok{(}\DataTypeTok{color =}\NormalTok{ Flagged_Y.N)) }\OperatorTok{+}
\StringTok{  }\KeywordTok{geom_abline}\NormalTok{(}\DataTypeTok{intercept =} \DecValTok{0}\NormalTok{, }\DataTypeTok{slope =} \DecValTok{1}\NormalTok{, }\DataTypeTok{color =} \StringTok{"blue"}\NormalTok{) }\OperatorTok{+}
\StringTok{  }\KeywordTok{xlim}\NormalTok{(}\DecValTok{0}\NormalTok{,}\DecValTok{60}\NormalTok{) }\OperatorTok{+}\StringTok{ }\KeywordTok{ylim}\NormalTok{(}\DecValTok{0}\NormalTok{, }\DecValTok{60}\NormalTok{) }\OperatorTok{+}
\StringTok{  }\KeywordTok{theme_bw}\NormalTok{()}
\end{Highlighting}
\end{Shaded}

\begin{verbatim}
## Warning: Removed 27 rows containing missing values (geom_point).
\end{verbatim}

\includegraphics{ASSP_CPUE_metadata_QAQC_files/figure-latex/unnamed-chunk-12-1.pdf}
.

\begin{quote}
The plot above shows the total number of ASSP vs.~total number of ASSP
before the standard ending. Blue line = slope of 1. Here we double check
that the standardized number of ASSP is always equal to or less than the
total number. Red points = data that has been flagged due to
inconsistencies in data entry. It does not appear that the reason these
entries were flagged effects the number of birds caught.
\end{quote}

\hypertarget{timing-of-assp-catches}{%
\subsection{Timing of ASSP catches}\label{timing-of-assp-catches}}

\begin{quote}
Next lets explore the frequency of catches in relation to the standard
ending. Do catches start dropping off before 5.3 hours after sunset?
After? NOTE - the timing of net closures will effect the number of late
night captures
\end{quote}

\begin{Shaded}
\begin{Highlighting}[]
\KeywordTok{ggplot}\NormalTok{(catches, }\KeywordTok{aes}\NormalTok{(catchPastSS)) }\OperatorTok{+}
\StringTok{  }\KeywordTok{geom_histogram}\NormalTok{(}\DataTypeTok{binwidth =} \DecValTok{10}\NormalTok{) }\OperatorTok{+}
\StringTok{  }\KeywordTok{geom_vline}\NormalTok{(}\DataTypeTok{xintercept =} \DecValTok{318}\NormalTok{, }\DataTypeTok{color =} \StringTok{"red"}\NormalTok{) }\OperatorTok{+}
\StringTok{  }\KeywordTok{xlab}\NormalTok{(}\StringTok{"Time past Sunset (min)"}\NormalTok{) }\OperatorTok{+}\StringTok{ }\KeywordTok{ylab}\NormalTok{(}\StringTok{"Number of ASSP Catches"}\NormalTok{) }\OperatorTok{+}
\StringTok{  }\KeywordTok{theme_bw}\NormalTok{()}
\end{Highlighting}
\end{Shaded}

\includegraphics{ASSP_CPUE_metadata_QAQC_files/figure-latex/unnamed-chunk-13-1.pdf}

\begin{quote}
5.3 hour cutoff occurs when number of catches are still relatively high.
How does that differ across years? In recent years (2014 - 2018) nets
were usually closed at 2am. In earlier years nets were sometimes left
open past 2am.
\end{quote}

\begin{Shaded}
\begin{Highlighting}[]
\KeywordTok{ggplot}\NormalTok{(catches, }\KeywordTok{aes}\NormalTok{(catchPastSS)) }\OperatorTok{+}
\StringTok{  }\KeywordTok{geom_histogram}\NormalTok{(}\DataTypeTok{binwidth =} \DecValTok{10}\NormalTok{) }\OperatorTok{+}
\StringTok{  }\KeywordTok{geom_vline}\NormalTok{(}\DataTypeTok{xintercept =} \DecValTok{318}\NormalTok{, }\DataTypeTok{color =} \StringTok{"red"}\NormalTok{) }\OperatorTok{+}
\StringTok{  }\KeywordTok{xlab}\NormalTok{(}\StringTok{"Time past Sunset (min)"}\NormalTok{) }\OperatorTok{+}\StringTok{ }\KeywordTok{ylab}\NormalTok{(}\StringTok{"Number of ASSP Catches"}\NormalTok{) }\OperatorTok{+}
\StringTok{  }\KeywordTok{facet_wrap}\NormalTok{(.}\OperatorTok{~}\StringTok{ }\NormalTok{year, }\DataTypeTok{scales =} \StringTok{"free"}\NormalTok{) }\OperatorTok{+}
\StringTok{  }\KeywordTok{theme_bw}\NormalTok{()}
\end{Highlighting}
\end{Shaded}

\includegraphics{ASSP_CPUE_metadata_QAQC_files/figure-latex/unnamed-chunk-14-1.pdf}

\begin{Shaded}
\begin{Highlighting}[]
\CommentTok{# }\AlertTok{NOTE}\CommentTok{: scales on x- and y- axis differ between panes}
\end{Highlighting}
\end{Shaded}

\begin{quote}
No clear pattern of catch timing can be seen across years.
\end{quote}

\begin{quote}
How does catch time post sunset differ between netting locations?
\end{quote}

\begin{Shaded}
\begin{Highlighting}[]
\KeywordTok{ggplot}\NormalTok{(catches, }\KeywordTok{aes}\NormalTok{(catchPastSS)) }\OperatorTok{+}
\StringTok{  }\KeywordTok{geom_histogram}\NormalTok{() }\OperatorTok{+}
\StringTok{  }\KeywordTok{geom_vline}\NormalTok{(}\DataTypeTok{xintercept =} \DecValTok{318}\NormalTok{, }\DataTypeTok{color =} \StringTok{"red"}\NormalTok{) }\OperatorTok{+}
\StringTok{  }\KeywordTok{xlab}\NormalTok{(}\StringTok{"Time past Sunset (min)"}\NormalTok{) }\OperatorTok{+}\StringTok{ }\KeywordTok{ylab}\NormalTok{(}\StringTok{"Number of Catches"}\NormalTok{) }\OperatorTok{+}
\StringTok{  }\KeywordTok{facet_wrap}\NormalTok{(.}\OperatorTok{~}\StringTok{ }\NormalTok{Location) }\OperatorTok{+}
\StringTok{  }\KeywordTok{theme_bw}\NormalTok{()}
\end{Highlighting}
\end{Shaded}

\begin{verbatim}
## `stat_bin()` using `bins = 30`. Pick better value with `binwidth`.
\end{verbatim}

\includegraphics{ASSP_CPUE_metadata_QAQC_files/figure-latex/unnamed-chunk-15-1.pdf}

\begin{quote}
There doesn't appear to be any distinct pattern in timing of catches
between locations, especially given the variation in sample sizes
between locations.
\end{quote}

\hypertarget{cpue}{%
\section{CPUE}\label{cpue}}

\hypertarget{visualization-of-cpue-per-session}{%
\subsection{visualization of CPUE per
session}\label{visualization-of-cpue-per-session}}

\begin{Shaded}
\begin{Highlighting}[]
\KeywordTok{ggplot}\NormalTok{(metadata, }\KeywordTok{aes}\NormalTok{(CPUEraw)) }\OperatorTok{+}
\StringTok{  }\KeywordTok{geom_histogram}\NormalTok{(}\DataTypeTok{binwidth =} \FloatTok{0.01}\NormalTok{) }\OperatorTok{+}
\StringTok{  }\KeywordTok{theme_bw}\NormalTok{()}
\end{Highlighting}
\end{Shaded}

\begin{verbatim}
## Warning: Removed 46 rows containing non-finite values (stat_bin).
\end{verbatim}

\includegraphics{ASSP_CPUE_metadata_QAQC_files/figure-latex/unnamed-chunk-16-1.pdf}

\begin{Shaded}
\begin{Highlighting}[]
\CommentTok{# summary of CPUE per session}
\KeywordTok{summary}\NormalTok{(metadata}\OperatorTok{$}\NormalTok{CPUEraw)}
\end{Highlighting}
\end{Shaded}

\begin{verbatim}
##    Min. 1st Qu.  Median    Mean 3rd Qu.    Max.    NA's 
## 0.00282 0.03245 0.05298 0.06270 0.08795 0.20779      46
\end{verbatim}

\begin{quote}
The graph and summary stats above show the distribution of
catch-per-unit-effort (ASSP/min).
\end{quote}

\hypertarget{visualization-of-cpue-per-standardized-session}{%
\subsection{visualization of CPUE per standardized
session}\label{visualization-of-cpue-per-standardized-session}}

\begin{Shaded}
\begin{Highlighting}[]
\KeywordTok{ggplot}\NormalTok{(metadata, }\KeywordTok{aes}\NormalTok{(CPUEstd)) }\OperatorTok{+}
\StringTok{  }\KeywordTok{geom_histogram}\NormalTok{(}\DataTypeTok{binwidth =} \FloatTok{0.01}\NormalTok{) }\OperatorTok{+}
\StringTok{  }\KeywordTok{theme_bw}\NormalTok{()}
\end{Highlighting}
\end{Shaded}

\begin{verbatim}
## Warning: Removed 47 rows containing non-finite values (stat_bin).
\end{verbatim}

\includegraphics{ASSP_CPUE_metadata_QAQC_files/figure-latex/unnamed-chunk-17-1.pdf}

\begin{Shaded}
\begin{Highlighting}[]
\CommentTok{# summary of CPUE per standardized session}
\KeywordTok{summary}\NormalTok{(metadata}\OperatorTok{$}\NormalTok{CPUEstd)}
\end{Highlighting}
\end{Shaded}

\begin{verbatim}
##    Min. 1st Qu.  Median    Mean 3rd Qu.    Max.    NA's 
## 0.00000 0.02858 0.05351 0.06161 0.08511 0.25810      47
\end{verbatim}

\begin{quote}
The graph and summary stats above show the distribution of standardized
catch-per-unit-effort (ASSPstd/min\_std). This distribution is more
constrained than the previous one, which is what we would expect with
the standard ending cutoff.
\end{quote}

\hypertarget{comparision-of-cpue-vs-cpuestd}{%
\subsection{comparision of CPUE vs
CPUEstd}\label{comparision-of-cpue-vs-cpuestd}}

\begin{Shaded}
\begin{Highlighting}[]
\KeywordTok{ggplot}\NormalTok{(metadata, }\KeywordTok{aes}\NormalTok{(CPUEraw, CPUEstd)) }\OperatorTok{+}
\StringTok{  }\KeywordTok{geom_point}\NormalTok{(}\KeywordTok{aes}\NormalTok{(}\DataTypeTok{color =}\NormalTok{ Flagged_Y.N)) }\OperatorTok{+}
\StringTok{  }\KeywordTok{geom_abline}\NormalTok{(}\DataTypeTok{intercept =} \DecValTok{0}\NormalTok{, }\DataTypeTok{slope =} \DecValTok{1}\NormalTok{, }\DataTypeTok{color =} \StringTok{"blue"}\NormalTok{) }\OperatorTok{+}
\StringTok{  }\KeywordTok{theme_bw}\NormalTok{()}
\end{Highlighting}
\end{Shaded}

\begin{verbatim}
## Warning: Removed 47 rows containing missing values (geom_point).
\end{verbatim}

\includegraphics{ASSP_CPUE_metadata_QAQC_files/figure-latex/unnamed-chunk-18-1.pdf}
.

\begin{quote}
The above graph explores the correlation between CPUE and CPUE std. Blue
line = slope of 1. As expected, the correlation is often 1:1, but with
variation as the number of ASSP caught and number of mistnetting minutes
were both effected by the standard ending cutoff but not always in a
proportional way. Red points = data that has been flagged due to
inconsistencies in data entry. It does not appear that the reason these
entries were flagged effects the overall CPUE. The three outliers on the
upper righthand side of the graph were checked to make sure the data was
accurate. Sure enough, these were nights with high numbers of ASSP
caught, but no errors in the data
\end{quote}

\hypertarget{what-other-variables-could-influence-cpue}{%
\subsection{What other variables could influence
CPUE?}\label{what-other-variables-could-influence-cpue}}

\hypertarget{net-dimensions}{%
\subsubsection{net dimensions}\label{net-dimensions}}

\begin{Shaded}
\begin{Highlighting}[]
\KeywordTok{ggplot}\NormalTok{(metadata, }\KeywordTok{aes}\NormalTok{(Net_dim, CPUEstd)) }\OperatorTok{+}
\StringTok{  }\KeywordTok{geom_boxplot}\NormalTok{() }\OperatorTok{+}
\StringTok{  }\KeywordTok{geom_text}\NormalTok{(}\DataTypeTok{data =}\NormalTok{ netUse,}
            \KeywordTok{aes}\NormalTok{(Net_dim, }\OtherTok{Inf}\NormalTok{, }\DataTypeTok{label =}\NormalTok{ n), }\DataTypeTok{vjust =} \DecValTok{1}\NormalTok{) }\OperatorTok{+}
\StringTok{  }\KeywordTok{theme_bw}\NormalTok{()}
\end{Highlighting}
\end{Shaded}

\begin{verbatim}
## Warning: Removed 47 rows containing non-finite values (stat_boxplot).
\end{verbatim}

\includegraphics{ASSP_CPUE_metadata_QAQC_files/figure-latex/unnamed-chunk-19-1.pdf}
.

\begin{quote}
Here is the frequency of standardized CPUE values broken up by the
dimensions of the net. The number above each box plot = the sample size.
Unknown net sizes make it hard to determine if the size of the net
influenced catch rates.
\end{quote}

\hypertarget{brood-patch-and-assumed-breeders}{%
\section{Brood Patch and Assumed
Breeders}\label{brood-patch-and-assumed-breeders}}

\begin{quote}
Below we explore the relationship between ASSP assumed to be breeding
and the time of year. ASSP were assumed to be breeding if they had a
bare broodpatch (brood patch score of 2-4; Ainley et al.~1990)
\end{quote}

\begin{Shaded}
\begin{Highlighting}[]
\NormalTok{monthCatches <-}\StringTok{ }\NormalTok{metadata }\OperatorTok\StringTok{ }
\StringTok{  }\KeywordTok{group_by}\NormalTok{(month) }\OperatorTok\StringTok{ }
\StringTok{  }\KeywordTok{tally}\NormalTok{()}

\KeywordTok{ggplot}\NormalTok{(metadata, }\KeywordTok{aes}\NormalTok{(month, BPfreq_Y)) }\OperatorTok{+}
\StringTok{  }\KeywordTok{geom_boxplot}\NormalTok{() }\OperatorTok{+}
\StringTok{  }\KeywordTok{ylab}\NormalTok{(}\StringTok{"Frequency of Assumed Breeders"}\NormalTok{) }\OperatorTok{+}
\StringTok{  }\KeywordTok{geom_text}\NormalTok{(}\DataTypeTok{data =}\NormalTok{ monthCatches,}
            \KeywordTok{aes}\NormalTok{(month, }\OtherTok{Inf}\NormalTok{, }\DataTypeTok{label =}\NormalTok{ n), }\DataTypeTok{vjust =} \DecValTok{1}\NormalTok{) }\OperatorTok{+}
\StringTok{  }\KeywordTok{theme_bw}\NormalTok{()}
\end{Highlighting}
\end{Shaded}

\begin{verbatim}
## Warning: Removed 27 rows containing non-finite values (stat_boxplot).
\end{verbatim}

\includegraphics{ASSP_CPUE_metadata_QAQC_files/figure-latex/unnamed-chunk-20-1.pdf}
.

\begin{quote}
The above graph shows frequency of assumed breeders (brood patch 2-4) in
relation to the month of netting effort. The number above each box plot
= sample size. The number of assumed breeders appears to be higher later
in the breeding season.
\end{quote}

\begin{Shaded}
\begin{Highlighting}[]
\KeywordTok{ggplot}\NormalTok{(metadata, }\KeywordTok{aes}\NormalTok{(month, BPfreq_Y)) }\OperatorTok{+}
\StringTok{  }\KeywordTok{geom_point}\NormalTok{(}\DataTypeTok{position =} \StringTok{"jitter"}\NormalTok{) }\OperatorTok{+}
\StringTok{  }\KeywordTok{ylab}\NormalTok{(}\StringTok{"Frequency of Assumed Breeders"}\NormalTok{) }\OperatorTok{+}
\StringTok{  }\CommentTok{# scale_color_gradient(low="blue", high="yellow") +}
\StringTok{  }\KeywordTok{facet_wrap}\NormalTok{(.}\OperatorTok{~}\NormalTok{year) }\OperatorTok{+}
\StringTok{  }\KeywordTok{theme_bw}\NormalTok{()}
\end{Highlighting}
\end{Shaded}

\begin{verbatim}
## Warning: Removed 27 rows containing missing values (geom_point).
\end{verbatim}

\includegraphics{ASSP_CPUE_metadata_QAQC_files/figure-latex/unnamed-chunk-21-1.pdf}
.

\begin{quote}
The above graph shows the frequency of assumed breeders caught in
mistnetting sessions across months, broken up by years. No distinct
patterns appear here.
\end{quote}

\end{document}
