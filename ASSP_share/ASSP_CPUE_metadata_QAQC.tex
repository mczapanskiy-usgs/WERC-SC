\PassOptionsToPackage{unicode=true}{hyperref} % options for packages loaded elsewhere
\PassOptionsToPackage{hyphens}{url}
%
\documentclass[
]{article}
\usepackage{lmodern}
\usepackage{amssymb,amsmath}
\usepackage{ifxetex,ifluatex}
\ifnum 0\ifxetex 1\fi\ifluatex 1\fi=0 % if pdftex
  \usepackage[T1]{fontenc}
  \usepackage[utf8]{inputenc}
  \usepackage{textcomp} % provides euro and other symbols
\else % if luatex or xelatex
  \usepackage{unicode-math}
  \defaultfontfeatures{Scale=MatchLowercase}
  \defaultfontfeatures[\rmfamily]{Ligatures=TeX,Scale=1}
\fi
% use upquote if available, for straight quotes in verbatim environments
\IfFileExists{upquote.sty}{\usepackage{upquote}}{}
\IfFileExists{microtype.sty}{% use microtype if available
  \usepackage[]{microtype}
  \UseMicrotypeSet[protrusion]{basicmath} % disable protrusion for tt fonts
}{}
\makeatletter
\@ifundefined{KOMAClassName}{% if non-KOMA class
  \IfFileExists{parskip.sty}{%
    \usepackage{parskip}
  }{% else
    \setlength{\parindent}{0pt}
    \setlength{\parskip}{6pt plus 2pt minus 1pt}}
}{% if KOMA class
  \KOMAoptions{parskip=half}}
\makeatother
\usepackage{xcolor}
\IfFileExists{xurl.sty}{\usepackage{xurl}}{} % add URL line breaks if available
\IfFileExists{bookmark.sty}{\usepackage{bookmark}}{\usepackage{hyperref}}
\hypersetup{
  pdftitle={QAQC of the CINP ASSP 1994 - 2018 CPUE data},
  pdfauthor={EKelsey},
  pdfborder={0 0 0},
  breaklinks=true}
\urlstyle{same}  % don't use monospace font for urls
\usepackage[margin=1in]{geometry}
\usepackage{graphicx,grffile}
\makeatletter
\def\maxwidth{\ifdim\Gin@nat@width>\linewidth\linewidth\else\Gin@nat@width\fi}
\def\maxheight{\ifdim\Gin@nat@height>\textheight\textheight\else\Gin@nat@height\fi}
\makeatother
% Scale images if necessary, so that they will not overflow the page
% margins by default, and it is still possible to overwrite the defaults
% using explicit options in \includegraphics[width, height, ...]{}
\setkeys{Gin}{width=\maxwidth,height=\maxheight,keepaspectratio}
\setlength{\emergencystretch}{3em}  % prevent overfull lines
\providecommand{\tightlist}{%
  \setlength{\itemsep}{0pt}\setlength{\parskip}{0pt}}
\setcounter{secnumdepth}{-2}
% Redefines (sub)paragraphs to behave more like sections
\ifx\paragraph\undefined\else
  \let\oldparagraph\paragraph
  \renewcommand{\paragraph}[1]{\oldparagraph{#1}\mbox{}}
\fi
\ifx\subparagraph\undefined\else
  \let\oldsubparagraph\subparagraph
  \renewcommand{\subparagraph}[1]{\oldsubparagraph{#1}\mbox{}}
\fi

% set default figure placement to htbp
\makeatletter
\def\fps@figure{htbp}
\makeatother


\title{QAQC of the CINP ASSP 1994 - 2018 CPUE data}
\author{EKelsey}
\date{4/10/2020}

\begin{document}
\maketitle

Load libraries

Load data and add columns neccessary for QAQC:

\hypertarget{time-and-mistnetting-effort}{%
\section{Time and Mistnetting
Effort}\label{time-and-mistnetting-effort}}

\hypertarget{graphical-check-of-app_sunset}{%
\subsection{Graphical check of
App\_sunset}\label{graphical-check-of-app_sunset}}

.

\begin{quote}
This graph shows the time of apparent sunset for netting sessions each
month. The range and timing for that time of year is as we would expect.
Thus we conclude that the suncalc function was used effectively to get
the sunset times associated with each mistnetting session.
\end{quote}

\hypertarget{graphical-check-of-std_ending}{%
\subsection{Graphical check of
Std\_ending}\label{graphical-check-of-std_ending}}

\hypertarget{plotted-by-month}{%
\subsubsection{Plotted by month}\label{plotted-by-month}}

.

\begin{quote}
This graph shows the time of standard ending (5.3 hours after sunset)
for netting sessions each month. The range and timing for these ending
track with sunset time as we would expect.
\end{quote}

\hypertarget{summarize-net_open-and-net_close}{%
\section{Summarize net\_open and
net\_close}\label{summarize-net_open-and-net_close}}

\begin{quote}
This is not a perfect way to summarize net open and close times because
the ``summarize'' function doesn't recognize times across midnight here.
But, by looking at the median and mean, we can tell that net open and
close times are usually what we would expect, with a few late/early
nights thrown in.
\end{quote}

\hypertarget{total-mistnetting-minutes-per-session}{%
\subsection{Total mistnetting minutes per
session}\label{total-mistnetting-minutes-per-session}}

\begin{quote}
Here we visualize the total number of minutes calcuated for each netting
session. We want to check that minutes were added correctly across
multiple open/close sessions and also that minutes were added accurately
across midnight. It looks like minutes were not added accurately across
midnight on four occasions (the outliers)
\end{quote}

\hypertarget{total-mistnetting-standard-minutes-per-session}{%
\subsection{Total mistnetting standard minutes per
session}\label{total-mistnetting-standard-minutes-per-session}}

\hypertarget{from-start-until-end-or-standard-ending-whichever-came-first}{%
\subsubsection{from start until end or standard ending, whichever came
first}\label{from-start-until-end-or-standard-ending-whichever-came-first}}

\begin{quote}
Here we visualise the total number of mintues for the standardized
session (from net open to net close or standard ending {[}5.3 hours
after sunset{]} whichever came first). The standardized minutes cut out
the erroneous minute calculations, as hoped. Also the max number of
standardized minutes is 317, which makes sense as the maximum amount of
time between sunset and 5.3 hours after.
\end{quote}

\hypertarget{compare-min-vs.-min_std-for-each-session}{%
\subsection{Compare min vs.~min\_std for each
session}\label{compare-min-vs.-min_std-for-each-session}}

.

\begin{quote}
This plot of minutes vs.~standardized mintues (before the 5.3 hour
standardized ending). Blue line = slope of 1. Here we can make sure that
minutes standardized is always equal or less than total minutes and
total number of standardized minutes isn't \textgreater{}317, indicating
that the 5.3 hour cutoff was applied.
\end{quote}

\hypertarget{assp}{%
\section{ASSP}\label{assp}}

\hypertarget{histogram-of-total-assp-caught-per-session}{%
\subsection{Histogram of total ASSP caught per
session}\label{histogram-of-total-assp-caught-per-session}}

\begin{quote}
This graph and summary stats show the distribution of total numbers of
ASSP caught per session.
\end{quote}

\hypertarget{histogram-of-total-assp-caught-per-standardized-session}{%
\subsection{Histogram of total ASSP caught per standardized
session}\label{histogram-of-total-assp-caught-per-standardized-session}}

\begin{quote}
This graph and summary stats show the distribution of total numbers of
ASSP caught before standard ending or net close, whichever came first.
This distribution is more constrained than the one above, which is what
we would expect with the standard ending cutoff.
\end{quote}

\hypertarget{comparison-of-assp-vs-asspstd}{%
\subsection{comparison of ASSP vs
ASSPstd}\label{comparison-of-assp-vs-asspstd}}

.

\begin{quote}
This plot of total number of ASSP vs.~total number of ASSP before the
standard ending. Blue line = slope of 1. Here we double check that the
standardized number of ASSP is always equal to or less than the total
number.
\end{quote}

\hypertarget{cpue}{%
\section{CPUE}\label{cpue}}

\hypertarget{visualization-of-cpue-per-session}{%
\subsection{visualization of CPUE per
session}\label{visualization-of-cpue-per-session}}

\begin{quote}
This graph and summary stats show the distribution of
catch-per-unit-effort (ASSP/min).
\end{quote}

\hypertarget{visualization-of-cpue-per-standardized-session}{%
\subsection{visualization of CPUE per standardized
session}\label{visualization-of-cpue-per-standardized-session}}

\begin{quote}
This graph and summary stats show the distribution of standardized
catch-per-unit-effort (ASSPstd/min\_std). This distribution is more
constrained than the one above, which is what we would expect with the
standard ending cutoff.
\end{quote}

\hypertarget{comparision-of-cpue-vs-cpuestd}{%
\subsection{comparision of CPUE vs
CPUEstd}\label{comparision-of-cpue-vs-cpuestd}}

.

\begin{quote}
This graph explores the correlation between CPUE and CPUE std. Blue line
= slope of 1. As expected, the correlation is often 1:1, but with
variation as the number of ASSP caught and number of mistnetting minutes
were both effected by the standard ending cutoff but not always in a
proportional way. The three outliers on the upper righthand side of the
graph were checked to make sure the data was accurate. Sure enough,
these were nights with high numbers of ASSP caught, but no errors in the
data
\end{quote}

\end{document}
